%%%%%%%%%%%%%%%%%%%%%%%%%%%%%%%%%%%%%%%%%%%%%%%%%%%%%%%%%%%%%%%%%%%%%%%%%%%%%%%
% NAME:             designdoc.tex
%
% AUTHOR:           Ethan D. Twardy <edtwardy@mtu.edu>
%
% DESCRIPTION:      Design Document for the project.
%
% CREATED:          05/15/2019
%
% LAST EDITED:      08/07/2019
%%%

\documentclass{designdoc}

\begin{document}
\section{Introduction}
\textit{SortMedia} is a program used to enumerate and restructure a Music
library. Taking as an argument the path of a directory to organize as a music
library, the application recursively searches the directory for media files,
parsing each one for its metadata, then organizing the files into a
(potentially) new directory structure according to some basic rules.

The needs of this kind of program have a tendency to change frequently, so the
goal of this design is to be as versatile as possible.

\section{Axes of Change}
An \textit{Axis of Change} for software is a reason for the code to undergo
some change as a result of a requirements change. It's often useful to predict
the axes of change that a project has before beginning its design. The
SortMedia project will be designed with the following axes of change in mind:

\subsection{Adding New Organizational Schemas}
An Organizational Schema is a set of rules that dictate the organization of
media files on the filesystem, as well as the metadata that they contain. The
schema that I prefer has changed slightly over the years, and is likely to
change again.

\subsection{Adding New Media File Types}
There are many types of media files. It's not reasonable to support all of them
from the beginning. The design shall make it easy to incorporate new media file
types into the project.

\subsection{Adding New File Operations to an Organizational Schema}
Separate from the organizational schema are the file operations. These are
rules that dictate how a change is performed to a file. With new Organizational
Schemas naturally comes the need for new File Operations.
\end{document}

%%%%%%%%%%%%%%%%%%%%%%%%%%%%%%%%%%%%%%%%%%%%%%%%%%%%%%%%%%%%%%%%%%%%%%%%%%%%%%%
